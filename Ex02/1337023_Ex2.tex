\documentclass[a4paper,12pt]{article}
\usepackage{exercise}

\begin{document}

\firstpage{02}{Matthias Bock}{1337023}

\section{Lab Questions}

\subsection{Connect the two programs}

\subsubsection{Client code}
\img{images/Ex02-1_Client_code.png}

\subsubsection{Client terminal}
\img{images/Ex02-1_Client_terminal.png}


\subsection{Display the source IP address on the server}

\subsubsection{Server code}
\img{images/Ex02-2_Server_code.png}

\subsubsection{Server terminal}
\img{images/Ex02-2_Server_terminal.png}

\subsection{Show that the server can process messages from three clients}

\subsubsection{Server terminal}
\img{images/Ex02-3_Server_terminal.png}

\subsubsection{Client 1 terminal}
\img{images/Ex02-3_Client1_terminal.png}

\subsubsection{Client 2 terminal}
\img{images/Ex02-3_Client2_terminal.png}

\subsubsection{Client 3 terminal}
\img{images/Ex02-3_Client3_terminal.png}

\subsection{How many sockets do the clients and the server each have}
The server and client both have one socket each.

\subsection{Explain why the server can support multiple clients}
Because the server and client use UDP to communicate.
UDP is a connectionless protocol, which means that there no connection established between client and server.
The server just receives packets from any client and then replies with his response.

\subsection{Implement a sequence number}

\subsubsection{Client code}
\img{images/Ex02-6_Client_code.png}

\subsubsection{Server code}
\img{images/Ex02-6_Server_code.png}

\subsubsection{Client terminal}
\img{images/Ex02-6_Client_terminal.png}

\subsubsection{Server terminal}
\img{images/Ex02-6_Server_terminal.png}

\subsection{Implement RTT display}

\subsubsection{Client code}
\img{images/Ex02-7_Client_code.png}

\subsubsection{Server code}
\img{images/Ex02-7_Server_code.png}

\subsubsection{Client terminal}
\img{images/Ex02-7_Client_terminal.png}

\subsubsection{Server terminal}
\img{images/Ex02-7_Server_terminal.png}

\subsection{Send 5 messages and compare the RTT values}

\subsubsection{Client terminal}
\img[0.9]{images/Ex02-8_Client_terminal.png}

\subsubsection{Comparison of RTT times}
In this example, the client and server were running on the same PC.
Because of this, the round trip time is overall extremely small.
But even here it is visible that the RTT is fluctuating between requests.
This is probably because the software still has to send the data through the network stack of the pc.
There, it is not guaranteed that every packet is processed immediately.
If it were sent to another PC, the overall delay would be longer.
Furthermore, the delay variations would probably be larger because the packet has to pass through the network infrastructure to get to the other host.
Here it experiences the 4 components of delay (propagation delay, queuing delay, processing delay, transmission delay), which all add a certain randomness.


\section{Homework questions}

\subsection{How is the source port number of the client program assigned?}
The source port number of the client program is not explicitly assigned.
It gets assigned automatically by the operating system.

\subsection{How is the source port number of the server program assigned?}
The source port number of the server program is assigned using the \textbf{bind} method of the socket class.
\textit{serverSocket.bind(('', serverPort))}.

\subsection{Is it possible to use clientSocket.bind() method in the client program? Please explain.}
Yes, it is possible to use the \textit{clientSocket.bind()} int the client program.
It could be used like this: '\textit{clientSocket.bind(('', clientPort))}'.
This would assign a static port to the client program which will be used as the source port every time it sends data to the server.

\subsection{What are multiplexing and demultiplexing?}
Multiplexing describes enables one PC to send packets from multiple sockets by adding the transport header to the packet before passing the segments to the network layer.
Demultiplexing enables the PC to deliver segments to the correct socket when receiving them from the network layer.

\subsection{Explain the demultiplexing process of UDP (e.g. which parameters are used)}
When a IP datagram arrives at the host, the transport-layer segment is extracted from it.
The host then uses the destination IP address and the destination port to direct the segment to the appropriate socket.
This process is called demultiplexing.

\subsection{Describe possible techniques to provide reliable service over UDP.}
It is possible to add reliability to UPD by building it into the application itself.
For example by adding acknowledgement or a sequence number to the data transmitted via UDP.
These can then be used to initiate a retransmission of lost packages.

\subsection{What are the differences between IPv4 and IPv6?}
IPv6 implements a few fields that IPv4 does not have.
For example, the IPv6 header has a \textbf{priority} field, that allows for the identification of priority among datagrams.
It also has a \textbf{flow label} to identify datagrams that are in the same "flow".
Another change makes sure that the IPv6 header no longer an variable options field, but a fixed size.
Instead, it has a field called \textbf{next header}, that identifies the upper layer protocol used for the data.
IPv6 also does not have a checksum built into the header.

\subsection{What is the relation of UDP \& RTP? Please explain the fields provided by RTP.}
RDP runs on top of UPD and extends it.
RDP provides the following additional fields:
\begin{itemize}
    \item \textbf{payload type identification:}
	indicates the type of encoding used for the payload.
	If the sender changes the the encoding, he informs the receiver via this field.
    \item \textbf{packet sequence numbering:}
	Incremented by one for each RTP packet sent.
	This allows the receiver to detect packet loss and restore the packet sequence.
    \item \textbf{time stamping:}
	Contains the sampling instant of the first byte in the data packet.
	For audio, the timestamp clock increments by one for each sampling period.
    \item \textbf{SSRC field:}
	Every stream in an RTP session has a distinct SSRC field.
\end{itemize}

\end{document}