\documentclass[a4]{article}
\usepackage{exercise}

\begin{document}

\firstpage{03}{Bock Matthias}{1337023}

\section{Lab questions}

\subsection{Connect Server and client}

\subsubsection{Client code}
\img{img/Ex03-1_Client_code.png}

\subsubsection{Client terminal}
\img{img/Ex03-1_Client_terminal.png}

\subsubsection{Server code}
\img{img/Ex03-1_Server_code.png}

\subsubsection{Server terminal}
\img{img/Ex03-1_Server_terminal.png}

\subsection{Allow multiple clients}

\subsubsection{Client 1 terminal}
\img{img/Ex03-2_Client1_terminal.png}

\subsubsection{Client 2 terminal}
\img{img/Ex03-2_Client2_terminal.png}

\subsubsection{Explanation}
The server currently only accepts one connection from one client.
That is why client 1 can communicate with the server without issues.
When client 2 tried to connect, it got stuck because the server did not call the \textit{accept()} method again.

\subsubsection{Server code modified}
\img{img/Ex03-2_Server_code.png}

\subsubsection{Server terminal}
\img{img/Ex03-2_Server_terminal.png}
The server is now able to accept connections from both clients.

\subsection{Explanation of Exercise 2}
\begin{itemize}
    \item \textbf{How many processes and threads are created by the server program?} \newline
        There is still only one server process, but now it creates two threads to handle both connecting clients. \newline
        It will create a new thread for every client that connects to the server.
    \item \textbf{What are the IP address(es) and port number(s) of these process(es) and thread(s).} \newline
        The IP address of both the server and the client is the same because they are running on the same computer (163.143.11.43). \newline
        The client uses a random port chosen by the operating system. \newline
        The Server uses the welcoming port 1200 to accept connections and then opens a new port for every connection.
        However, the welcoming port number is always used as the port number advertised to the client.
    \item \textbf{In the beginning, what is problem with client-2? After modification, why can client-2 connect to the server?}
        At the beginning the server only accepted one connection and then enters a loop to receive data from the client. \newline
        This prevent another client from connecting because the \textit{accept()} method is not called again. \newline
        After the modification, the server now accepts clients an a loop, dispatching the data receiving to a new thread.
        That way, a new client can be accepted while the old one still sends data.
\end{itemize}

\subsection{Estimate RTT}

\subsubsection{Client code}
\img{img/Ex03-4_Client_code.png}

\subsubsection{Client terminal}
\img{img/Ex03-4_Client_terminal.png}

\subsubsection{Server code}
\img{img/Ex03-4_Server_code.png}

\subsection{Send data to two servers}

\subsubsection{Client code}
\img{img/Ex03-5_Client_code.png}

\subsubsection{Server 1 terminal}
\img{img/Ex03-5_Server1_terminal.png}

\subsubsection{Server 2 terminal}
\img{img/Ex03-5_Server2_terminal.png}

\subsubsection{Client terminal}
\img{img/Ex03-5_Client_terminal.png}

\subsection{Explain question 5}
TODO

\subsection{Bonus question}
TODO

\section{Homework questions}

\subsection{Suppose a server is serving ONLY five TCP clients, each has one connection. How many sockets does the server have at least? Please explain.}
\subsection{Suppose a server is serving ONLY five UDP clients. How many sockets should the server have at least? Please explain.}
\subsection{In TCP, what are the differences in the purposes of reliable data transfer, flow control, and congestion control?}
\subsection{Describe the steps when a client and a server close the TCP connection between them.}
\subsection{What are the differences between a process and a thread.}
\subsection{What is the meaning of Seq=92 and ACK=100 in the Fig. 1a?}
\subsection{In Fig. 1b, why does the second ACK segment have ACK=120?}
\subsection{In Fig. 1b, why does host A resend segment with Seq=92?}
\subsection{If a host receives three ACK segments with the same ACK values, what is its action?}
\subsection{How can a TCP sender detect a loss?}

\end{document}